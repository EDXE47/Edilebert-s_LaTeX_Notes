\documentclass[a4paper]{book}

\usepackage{edilnotes_1} % the premable is contained in note-book-format.tex for universal application
 
%%%%%%%%%%%%%%%%%%%%%%%%%%%%%%%%%%%%%%%%%
% Adapted from "The Legrand Orange Book"
% by Mathias Legrand (legrand.mathias@gmail.com)
% and Modifications by Vel (vel@latextemplates.com) 
% Structural Definitions File
% Version 2.1 (26/09/2018)
%
% License:
% CC BY-NC-SA 3.0 (http://creativecommons.org/licenses/by-nc-sa/3.0/)
%
%%%%%%%%%%%%%%%%%%%%%%%%%%%%%%%%%%%%%%%%%

%--------------------------------------------------------------
% EDILEBERT's PREAMBLE

\usepackage[margin=1in]{geometry}
\usepackage{verbatim}
\usepackage{listings}
\usepackage{pgf}
\usepackage{tikz}
\usepackage{physics}
\usepackage{circuitikz}
\usepackage{pgfplots}
\usepackage{mathtools}
\usepackage{hyperref}
\usepackage{tkz-fct}  
\usepackage{enumitem}
\usepackage{lastpage}
\usepackage{fancyhdr}
\usepackage{array}
%\usepackage{courier}
\usepackage{caption}
\usepackage{multicol}
\usepackage{epigraph}
\usepackage{subcaption}
\usepackage{pgfplots}
\usepackage{float}

\usepackage{wrapfig}
\usepackage{pifont}
\usepackage[makeroom]{cancel}

\setlength{\parindent}{0pt}

\usepackage{bm} % for \bm
\usepackage{fixmath} % for \mathbold
\usepackage{scalerel}
\newlength\bshft
\bshft=.18pt\relax
\def\fakebold#1{\ThisStyle{\ooalign{$\SavedStyle#1$\cr%
  \kern-\bshft$\SavedStyle#1$\cr%
  \kern\bshft$\SavedStyle#1$}}}

\usepackage{pgfpages}
\pgfdeclarelayer{bg}
\pgfsetlayers{bg,main}

\iffalse

\pgfpagesdeclarelayout{boxed}
{
  \edef\pgfpageoptionborder{0pt}
}
{
  \pgfpagesphysicalpageoptions
  {%
    logical pages=1,%
  }
  \pgfpageslogicalpageoptions{1}
  {
    border code=\pgfsetlinewidth{2pt}\pgfstroke,%
    border shrink=\pgfpageoptionborder,%
    resized width=.88\pgfphysicalwidth,%
    resized height=.88\pgfphysicalheight,%
    center=\pgfpoint{.5\pgfphysicalwidth}{.5\pgfphysicalheight}%
  }%
}

\fi

\usepgfplotslibrary{fillbetween}
\usetikzlibrary{intersections}

%PRE BETA BUILD :: 

%=========== STAT MACROS =========
\newcommand{\cov}{\mathrm{cov}}
\newcommand{\E}{\mathrm{E}}
\newcommand{\F}{\mathrm{F}}
\newcommand{\V}{\mathrm{V}}
\newcommand{\N}{\mathrm{N}}
\renewcommand{\S}{\mathrm{S}}
\newcommand{\Se}{\mathrm{Se}}
\newcommand{\Hyp}{\mathrm{H}}
\newcommand{\MS}{\mathrm{MS}}

\newcommand{\cmark}{\ding{52}}%
\newcommand{\xmark}{\ding{55}}%

\newcommand{\simly}{|||^\mathrm{ly}}

%=================================
\usepackage{mathtools}
\DeclarePairedDelimiter\ceil{\lceil}{\rceil}
\DeclarePairedDelimiter\floor{\lfloor}{\rfloor}


\usepackage{amsmath}
\usepackage{amssymb}
%\usepackage{MnSymbol} %for CUPDOT
%\DeclareSymbolFont{Symbols}{OMS}{cmsy}{m}{n}
%\SetSymbolFont{Symbols}{bold}{OMS}{cmsy}{b}{n}
%\DeclareMathSymbol{\Setminus}{\mathbin}{Symbols}{"6E}
\usepackage{amsfonts}
\usepackage{nicefrac}

\usepackage{amsthm}



\newtheorem{lemma}{Lemma}
\numberwithin{lemma}{subsubsection}	
\numberwithin{lemma}{subsection}	
\numberwithin{lemma}{section}

\iffalse

\newtheorem{theorem}{Theorem}[chapter]
\numberwithin{theorem}{subsubsection}	
\numberwithin{theorem}{subsection}	
\numberwithin{theorem}{section}

\newtheorem{definition}{Definition}[chapter]
\numberwithin{definition}{subsubsection}	
\numberwithin{definition}{subsection}	
\numberwithin{definition}{section}

\newtheorem{remark}{Remark}[chapter]
\numberwithin{remark}{subsubsection}	
\numberwithin{remark}{subsection}	
\numberwithin{remark}{section}



\theoremstyle{definition}
\newtheorem{eg}{Example}[chapter]
\numberwithin{eg}{subsubsection}	
\numberwithin{eg}{subsection}	
\numberwithin{eg}{section}

\fi



\newcommand\restr[2]{{% we make the whole thing an ordinary symbol
  \left.\kern-\nulldelimiterspace % automatically resize the bar with \right
  #1 % the function
  \vphantom{\big|} % pretend it's a little taller at normal size
  \right|_{#2} % this is the delimiter
  }}
  
\pgfplotsset{holdot/.style={fill=white,only marks,mark=*}}

\usepackage{marginnote}

\newcommand{\HRule}{\rule{\textwidth}{.4mm}}

\newcommand{\Section}[2]{\section{#1}
\vspace{-.45cm} \HRule \\ \vspace{-3pt} #2 \\[.5cm]}

\newcommand{\SEction}[2]{\section*{#1}
\vspace{-.45cm} \HRule \\[-6pt]}

%\newcommand{\SEction}[2]{\section*{#1}
%\vspace{-.6cm} \HRule \\ \hfill #2 \\[.5cm]}

\usepackage[scr = mathpi]{mathalfa}
\usepackage{mathrsfs}

\makeatletter
\newcommand\RedeclareMathOperator{%
  \@ifstar{\def\rmo@s{m}\rmo@redeclare}{\def\rmo@s{o}\rmo@redeclare}%
}
% this is taken from \renew@command
\newcommand\rmo@redeclare[2]{%
  \begingroup \escapechar\m@ne\xdef\@gtempa{{\string#1}}\endgroup
  \expandafter\@ifundefined\@gtempa
     {\@latex@error{\noexpand#1undefined}\@ehc}%
     \relax
  \expandafter\rmo@declmathop\rmo@s{#1}{#2}}
% This is just \@declmathop without \@ifdefinable
\newcommand\rmo@declmathop[3]{%
  \DeclareRobustCommand{#2}{\qopname\newmcodes@#1{#3}}%
}
\@onlypreamble\RedeclareMathOperator
\makeatother

\newcommand{\Part}[1]{\mathscr{P}[#1]}
\newcommand{\Riem}[1]{\mathscr{R}[#1]}
\newcommand{\Real}{\mathbb{R}}
\newcommand{\power}[1]{\mathcal{P}(#1)}

\newcommand{\sa}{\mathcal{A}}
\newcommand{\sm}{\mathcal{M}}
\newcommand{\sG}{\mathcal{G}}
\newcommand{\sF}{\mathcal{F}}

\newcommand{\If}{\text{ if }}

\newcommand{\upRint}[2]{
  \overline{\int_{#1}^{#2}}
}
\newcommand{\loRint}[2]{
  \underline{\int_{#1}^{#2}}
}
  
\renewcommand\qedsymbol{$\blacksquare$}  
  
\newcommand{\inv}{^{-1}}

%--------------------------------------------------------------

%----------------------------------------------------------------------------------------
%	VARIOUS REQUIRED PACKAGES AND CONFIGURATIONS
%----------------------------------------------------------------------------------------

\usepackage{graphicx} % Required for including pictures
\graphicspath{{Pictures/}} % Specifies the directory where pictures are stored

\usepackage{lipsum} % Inserts dummy text

\usepackage{tikz} % Required for drawing custom shapes

\usepackage[english]{babel} % English language/hyphenation

\usepackage{enumitem} % Customize lists
\setlist{nolistsep} % Reduce spacing between bullet points and numbered lists

\usepackage{booktabs} % Required for nicer horizontal rules in tables

\usepackage{xcolor} % Required for specifying colors by name
\definecolor{ocre}{RGB}{0,0,0} % Define the orange color used for highlighting throughout the book

%----------------------------------------------------------------------------------------
%	MARGINS
%----------------------------------------------------------------------------------------

\usepackage{geometry} % Required for adjusting page dimensions and margins

\geometry{
	paper=a4paper, % Paper size, change to letterpaper for US letter size
	top=3cm, % Top margin
	bottom=3cm, % Bottom margin
	left=3cm, % Left margin
	right=3cm, % Right margin
	headheight=14pt, % Header height
	footskip=1.4cm, % Space from the bottom margin to the baseline of the footer
	headsep=10pt, % Space from the top margin to the baseline of the header
	%showframe, % Uncomment to show how the type block is set on the page
}

%----------------------------------------------------------------------------------------
%	FONTS
%----------------------------------------------------------------------------------------

\usepackage{avant} % Use the Avantgarde font for headings
%\usepackage{times} % Use the Times font for headings
%\usepackage{mathptmx} % Use the Adobe Times Roman as the default text font together with math symbols from the Sym­bol, Chancery and Com­puter Modern fonts

\usepackage{microtype} % Slightly tweak font spacing for aesthetics
\usepackage[utf8]{inputenc} % Required for including letters with accents
\usepackage[T1]{fontenc} % Use 8-bit encoding that has 256 glyphs

%----------------------------------------------------------------------------------------
%	BIBLIOGRAPHY AND INDEX
%----------------------------------------------------------------------------------------

\usepackage[style=numeric,citestyle=numeric,sorting=nyt,sortcites=true,autopunct=true,babel=hyphen,hyperref=true,abbreviate=false,backref=true,backend=biber]{biblatex}
\addbibresource{bibliography.bib} % BibTeX bibliography file
\defbibheading{bibempty}{}

\usepackage{calc} % For simpler calculation - used for spacing the index letter headings correctly
\usepackage{makeidx} % Required to make an index
\makeindex % Tells LaTeX to create the files required for indexing

%----------------------------------------------------------------------------------------
%	MAIN TABLE OF CONTENTS
%----------------------------------------------------------------------------------------

\usepackage{titletoc} % Required for manipulating the table of contents

\iffalse{
\contentsmargin{0cm} % Removes the default margin

% Part text styling (this is mostly taken care of in the PART HEADINGS section of this file)
\titlecontents{part}
	[0cm] % Left indentation
	{\addvspace{20pt}\bfseries} % Spacing and font options for parts
	{}
	{}
	{}

% Chapter text styling
\titlecontents{chapter}
	[1.25cm] % Left indentation
	{\addvspace{12pt}\large\sffamily\bfseries} % Spacing and font options for chapters
	{\color{ocre!60}\contentslabel[\Large\thecontentslabel]{1.25cm}\color{ocre}} % Formatting of numbered sections of this type
	{\color{ocre}} % Formatting of numberless sections of this type
	{\color{ocre!60}\normalsize\;\titlerule*[.5pc]{.}\;\thecontentspage} % Formatting of the filler to the right of the heading and the page number

% Section text styling
\titlecontents{section}
	[1.25cm] % Left indentation
	{\addvspace{3pt}\sffamily\bfseries} % Spacing and font options for sections
	{\contentslabel[\thecontentslabel]{1.25cm}} % Formatting of numbered sections of this type
	{} % Formatting of numberless sections of this type
	{\hfill\color{black}\thecontentspage} % Formatting of the filler to the right of the heading and the page number

% Subsection text styling
\titlecontents{subsection}
	[1.25cm] % Left indentation
	{\addvspace{1pt}\sffamily\small} % Spacing and font options for subsections
	{\contentslabel[\thecontentslabel]{1.25cm}} % Formatting of numbered sections of this type
	{} % Formatting of numberless sections of this type
	{\ \titlerule*[.5pc]{.}\;\thecontentspage} % Formatting of the filler to the right of the heading and the page number

% Figure text styling
\titlecontents{figure}
	[1.25cm] % Left indentation
	{\addvspace{1pt}\sffamily\small} % Spacing and font options for figures
	{\thecontentslabel\hspace*{1em}} % Formatting of numbered sections of this type
	{} % Formatting of numberless sections of this type
	{\ \titlerule*[.5pc]{.}\;\thecontentspage} % Formatting of the filler to the right of the heading and the page number

% Table text styling
\titlecontents{table}
	[1.25cm] % Left indentation
	{\addvspace{1pt}\sffamily\small} % Spacing and font options for tables
	{\thecontentslabel\hspace*{1em}} % Formatting of numbered sections of this type
	{} % Formatting of numberless sections of this type
	{\ \titlerule*[.5pc]{.}\;\thecontentspage} % Formatting of the filler to the right of the heading and the page number

%----------------------------------------------------------------------------------------
%	MINI TABLE OF CONTENTS IN PART HEADS
%----------------------------------------------------------------------------------------

% Chapter text styling
\titlecontents{lchapter}
	[0em] % Left indentation
	{\addvspace{15pt}\large\sffamily\bfseries} % Spacing and font options for chapters
	{\color{ocre}\contentslabel[\Large\thecontentslabel]{1.25cm}\color{ocre}} % Chapter number
	{}  
	{\color{ocre}\normalsize\sffamily\bfseries\;\titlerule*[.5pc]{.}\;\thecontentspage} % Page number

 Section text styling
\titlecontents{lsection}
	[0em] % Left indentation
	{\sffamily\small} % Spacing and font options for sections
	{\contentslabel[\thecontentslabel]{1.25cm}} % Section number
	{}
	{}

% Subsection text styling (note these aren't shown by default, display them by searchings this file for tocdepth and reading the commented text)
\titlecontents{lsubsection}
	[.5em] % Left indentation
	{\sffamily\footnotesize} % Spacing and font options for subsections
	{\contentslabel[\thecontentslabel]{1.25cm}}
	{}
	{}
} \fi
%----------------------------------------------------------------------------------------
%	HEADERS AND FOOTERS
%----------------------------------------------------------------------------------------

\usepackage{fancyhdr} % Required for header and footer configuration

\pagestyle{fancy} % Enable the custom headers and footers

\renewcommand{\chaptermark}[1]{\markboth{\sffamily\normalsize\bfseries\chaptername\ \thechapter.\ #1}{}} % Styling for the current chapter in the header
\renewcommand{\sectionmark}[1]{\markright{\sffamily\normalsize\thesection\hspace{5pt}#1}{}} % Styling for the current section in the header

\fancyhf{} % Clear default headers and footers
\fancyhead[LE,RO]{\sffamily\scriptsize\thepage} % Styling for the page number in the header
\fancyhead[LO]{\scriptsize\rightmark} % Print the nearest section name on the left side of odd pages
\fancyhead[RE]{\scriptsize\leftmark} % Print the current chapter name on the right side of even pages
%\fancyfoot[C]{\thepage} % Uncomment to include a footer

\renewcommand{\headrulewidth}{0.5pt} % Thickness of the rule under the header

\fancypagestyle{plain}{% Style for when a plain pagestyle is specified
	\fancyhead{}\renewcommand{\headrulewidth}{0pt}%
}

% Removes the header from odd empty pages at the end of chapters
\makeatletter
\renewcommand{\cleardoublepage}{
\clearpage\ifodd\c@page\else
\hbox{}
\vspace*{\fill}
\thispagestyle{empty}
\newpage
\fi}

%----------------------------------------------------------------------------------------
%	THEOREM STYLES
%----------------------------------------------------------------------------------------

\usepackage{amsmath,amsfonts,amssymb,amsthm} % For math equations, theorems, symbols, etc

\newcommand{\intoo}[2]{\mathopen{]}#1\,;#2\mathclose{[}}
\newcommand{\ud}{\mathop{\mathrm{{}d}}\mathopen{}}
\newcommand{\intff}[2]{\mathopen{[}#1\,;#2\mathclose{]}}
\renewcommand{\qedsymbol}{$\blacksquare$}
\newtheorem{notation}{Notation}[chapter]

% Boxed/framed environments
\newtheoremstyle{ocrenumbox}% Theorem style name
{0pt}% Space above
{0pt}% Space below
{\normalfont}% Body font
{}% Indent amount
{\small\bf\sffamily\color{ocre}}% Theorem head font
{\;}% Punctuation after theorem head
{0.25em}% Space after theorem head
{\small\sffamily\color{ocre}\thmname{#1}\nobreakspace\thmnumber{\@ifnotempty{#1}{}\@upn{#2}}% Theorem text (e.g. Theorem 2.1)
\thmnote{\nobreakspace\the\thm@notefont\sffamily\bfseries\color{black}---\nobreakspace#3.}} % Optional theorem note

\newtheoremstyle{blacknumex}% Theorem style name
{5pt}% Space above
{5pt}% Space below
{\normalfont}% Body font
{} % Indent amount
{\small\bf\sffamily}% Theorem head font
{\;}% Punctuation after theorem head
{0.25em}% Space after theorem head
{\small\sffamily{\tiny\ensuremath{\blacksquare}}\nobreakspace\thmname{#1}\nobreakspace\thmnumber{\@ifnotempty{#1}{}\@upn{#2}}% Theorem text (e.g. Theorem 2.1)
\thmnote{\nobreakspace\the\thm@notefont\sffamily\bfseries---\nobreakspace#3.}}% Optional theorem note

\newtheoremstyle{blacknumbox} % Theorem style name
{0pt}% Space above
{0pt}% Space below
{\normalfont}% Body font
{}% Indent amount
{\small\bf\sffamily}% Theorem head font
{\;}% Punctuation after theorem head
{0.25em}% Space after theorem head
{\small\sffamily\thmname{#1}\nobreakspace\thmnumber{\@ifnotempty{#1}{}\@upn{#2}}% Theorem text (e.g. Theorem 2.1)
\thmnote{\nobreakspace\the\thm@notefont\sffamily\bfseries---\nobreakspace#3.}}% Optional theorem note

% Non-boxed/non-framed environments
\newtheoremstyle{ocrenum}% Theorem style name
{5pt}% Space above
{5pt}% Space below
{\normalfont}% Body font
{}% Indent amount
{\small\bf\sffamily\color{ocre}}% Theorem head font
{\;}% Punctuation after theorem head
{0.25em}% Space after theorem head
{\small\sffamily\color{ocre}\thmname{#1}\nobreakspace\thmnumber{\@ifnotempty{#1}{}\@upn{#2}}% Theorem text (e.g. Theorem 2.1)
\thmnote{\nobreakspace\the\thm@notefont\sffamily\bfseries\color{black}---\nobreakspace#3.}} % Optional theorem note
\makeatother

% Defines the theorem text style for each type of theorem to one of the three styles above
\newcounter{dummy} 
\numberwithin{dummy}{section}

\theoremstyle{ocrenumbox}
\newtheorem{theoremeT}[dummy]{Theorem}
\newtheorem{problem}{Problem}[chapter]
\newtheorem{exerciseT}{Exercise}[chapter]

\theoremstyle{blacknumex}
\newtheorem{exampleT}{Example}[chapter]

\theoremstyle{blacknumbox}
\newtheorem{vocabulary}{Vocabulary}[chapter]
\newtheorem{definitionT}{Definition}[section]
\newtheorem{remarkT}{Remark}[section]
\newtheorem{corollaryT}[dummy]{Corollary}

\theoremstyle{ocrenum}
\newtheorem{proposition}[dummy]{Proposition}

%----------------------------------------------------------------------------------------
%	DEFINITION OF COLORED BOXES
%----------------------------------------------------------------------------------------

\RequirePackage[framemethod=TikZ]{mdframed} % Required for creating the theorem, definition, exercise and corollary boxes

% Theorem box
\newmdenv[skipabove=7pt,
skipbelow=7pt,
backgroundcolor=black!3,
linecolor=ocre,
innerleftmargin=5pt,
innerrightmargin=5pt,
innertopmargin=5pt,
leftmargin=0cm,
rightmargin=0cm,
innerbottommargin=5pt,
roundcorner=5pt]{tBox}

% Exercise box	  
\newmdenv[skipabove=7pt,
skipbelow=7pt,
rightline=false,
leftline=true,
topline=false,
bottomline=false,
backgroundcolor=ocre!10,
linecolor=ocre,
innerleftmargin=5pt,
innerrightmargin=5pt,
innertopmargin=5pt,
innerbottommargin=5pt,
leftmargin=0cm,
rightmargin=0cm,
linewidth=4pt,
roundcorner=5pt]{eBox}	

% Definition box
\newmdenv[skipabove=7pt,
skipbelow=7pt,
rightline=false,
leftline=true,
topline=false,
bottomline=false,
linecolor=ocre,
innerleftmargin=5pt,
innerrightmargin=5pt,
innertopmargin=0pt,
leftmargin=0cm,
rightmargin=0cm,
linewidth=4pt,
innerbottommargin=0pt,
roundcorner=5pt]{dBox}	

% Corollary box
\newmdenv[skipabove=7pt,
skipbelow=7pt,
rightline=false,
leftline=true,
topline=false,
bottomline=false,
linecolor=gray,
backgroundcolor=black!3,
innerleftmargin=5pt,
innerrightmargin=5pt,
innertopmargin=5pt,
leftmargin=0cm,
rightmargin=0cm,
linewidth=4pt,
innerbottommargin=5pt,
roundcorner=5pt]{cBox}

% Creates an environment for each type of theorem and assigns it a theorem text style from the "Theorem Styles" section above and a colored box from above
\newenvironment{theorem}{\begin{tBox}\begin{theoremeT}}{\end{theoremeT}\end{tBox}}
\newenvironment{exercise}{\begin{eBox}\begin{exerciseT}}{\hfill{\color{ocre}\tiny\ensuremath{\blacksquare}}\end{exerciseT}\end{eBox}}				  
\newenvironment{definition}{\begin{dBox}\begin{definitionT}}{\end{definitionT}\end{dBox}}	
\newenvironment{remark}{\begin{dBox}\begin{remarkT}}{\end{remarkT}\end{dBox}}	
\newenvironment{example}{\begin{exampleT}}{\hfill{\tiny\ensuremath{\blacksquare}}\end{exampleT}}		
\newenvironment{corollary}{\begin{cBox}\begin{corollaryT}}{\end{corollaryT}\end{cBox}}	

%----------------------------------------------------------------------------------------
%	CLAIM and PROOF TITLE
%----------------------------------------------------------------------------------------


\newcommand{\claim}{{\small \sffamily \bfseries claim: }}
%\renewenvironment{proof}{{\sffamily \bfseries Proof}}{\qed}
\usepackage{xpatch}
\xpatchcmd{\proof}{\itshape}{\small \sffamily \bfseries}{}{}

%----------------------------------------------------------------------------------------
%	SECTION NUMBERING IN THE MARGIN
%----------------------------------------------------------------------------------------

\makeatletter
\renewcommand{\@seccntformat}[1]{\llap{\textcolor{ocre}{\csname the#1\endcsname}\hspace{1em}}}                    
\renewcommand{\section}{\@startsection{section}{1}{\z@}
{-4ex \@plus -1ex \@minus -.4ex}
{1ex \@plus.2ex }
{\normalfont\large\sffamily\bfseries}}
\renewcommand{\subsection}{\@startsection {subsection}{2}{\z@}
{-3ex \@plus -0.1ex \@minus -.4ex}
{0.5ex \@plus.2ex }
{\normalfont\sffamily\bfseries}}
\renewcommand{\subsubsection}{\@startsection {subsubsection}{3}{\z@}
{-2ex \@plus -0.1ex \@minus -.2ex}
{.2ex \@plus.2ex }
{\normalfont\small\sffamily\bfseries}}                        
\renewcommand\paragraph{\@startsection{paragraph}{4}{\z@}
{-2ex \@plus-.2ex \@minus .2ex}
{.1ex}
{\normalfont\small\sffamily\bfseries}}

%----------------------------------------------------------------------------------------
%	PART HEADINGS
%----------------------------------------------------------------------------------------

% Numbered part in the table of contents
\newcommand{\@mypartnumtocformat}[2]{%
	\setlength\fboxsep{0pt}%
	\noindent\colorbox{ocre!20}{\strut\parbox[c][.7cm]{\ecart}{\color{ocre!70}\Large\sffamily\bfseries\centering#1}}\hskip\esp\colorbox{ocre!40}{\strut\parbox[c][.7cm]{\linewidth-\ecart-\esp}{\Large\sffamily\centering#2}}%
}

% Unnumbered part in the table of contents
\newcommand{\@myparttocformat}[1]{%
	\setlength\fboxsep{0pt}%
	\noindent\colorbox{ocre!40}{\strut\parbox[c][.7cm]{\linewidth}{\Large\sffamily\centering#1}}%
}

\newlength\esp
\setlength\esp{4pt}
\newlength\ecart
\setlength\ecart{1.2cm-\esp}
\newcommand{\thepartimage}{}%
\newcommand{\partimage}[1]{\renewcommand{\thepartimage}{#1}}%
\def\@part[#1]#2{%
\ifnum \c@secnumdepth >-2\relax%
\refstepcounter{part}%
\addcontentsline{toc}{part}{\texorpdfstring{\protect\@mypartnumtocformat{\thepart}{#1}}{\partname~\thepart\ ---\ #1}}
\else%
\addcontentsline{toc}{part}{\texorpdfstring{\protect\@myparttocformat{#1}}{#1}}%
\fi%
\startcontents%
\markboth{}{}%
{\thispagestyle{empty}%
\begin{tikzpicture}[remember picture,overlay]%
\node at (current page.north west){\begin{tikzpicture}[remember picture,overlay]%	
\fill[ocre!20](0cm,0cm) rectangle (\paperwidth,-\paperheight);
\node[anchor=north] at (4cm,-3.25cm){\color{ocre!40}\fontsize{220}{100}\sffamily\bfseries\thepart}; 
\node[anchor=south east] at (\paperwidth-1cm,-\paperheight+1cm){\parbox[t][][t]{8.5cm}{
\printcontents{l}{0}{\setcounter{tocdepth}{1}}% The depth to which the Part mini table of contents displays headings; 0 for chapters only, 1 for chapters and sections and 2 for chapters, sections and subsections
}};
\node[anchor=north east] at (\paperwidth-1.5cm,-3.25cm){\parbox[t][][t]{15cm}{\strut\raggedleft\color{white}\fontsize{30}{30}\sffamily\bfseries#2}};
\end{tikzpicture}};
\end{tikzpicture}}%
\@endpart}
\def\@spart#1{%
\startcontents%
\phantomsection
{\thispagestyle{empty}%
\begin{tikzpicture}[remember picture,overlay]%
\node at (current page.north west){\begin{tikzpicture}[remember picture,overlay]%	
\fill[ocre!20](0cm,0cm) rectangle (\paperwidth,-\paperheight);
\node[anchor=north east] at (\paperwidth-1.5cm,-3.25cm){\parbox[t][][t]{15cm}{\strut\raggedleft\color{white}\fontsize{30}{30}\sffamily\bfseries#1}};
\end{tikzpicture}};
\end{tikzpicture}}
\addcontentsline{toc}{part}{\texorpdfstring{%
\setlength\fboxsep{0pt}%
\noindent\protect\colorbox{ocre!40}{\strut\protect\parbox[c][.7cm]{\linewidth}{\Large\sffamily\protect\centering #1\quad\mbox{}}}}{#1}}%
\@endpart}
\def\@endpart{\vfil\newpage
\if@twoside
\if@openright
\null
\thispagestyle{empty}%
\newpage
\fi
\fi
\if@tempswa
\twocolumn
\fi}

%----------------------------------------------------------------------------------------
%	CHAPTER HEADINGS
%----------------------------------------------------------------------------------------

% A switch to conditionally include a picture, implemented by Christian Hupfer


\newif\ifusechapterimage
\usechapterimagetrue
\newcommand{\thechapterimage}{}%
\newcommand{\chapterimage}[1]{\ifusechapterimage\renewcommand{\thechapterimage}{#1}\fi}%
\newcommand{\autodot}{.}
%\def\@makechapterhead#1{%
%{\parindent \z@ \raggedright \normalfont
%\ifnum \c@secnumdepth >\m@ne
%\if@mainmatter
%\begin{tikzpicture}[remember picture,overlay]
%\node at (current page.north west)
%{\begin{tikzpicture}[remember picture,overlay]
%\node[anchor=north west,inner sep=0pt] at (0,0) {\ifusechapterimage\includegraphics[width=\paperwidth]{\thechapterimage}\fi};
%\draw[anchor=west] (\Gm@lmargin,-9cm) node [line width=2pt,rounded corners=15pt,draw=ocre,fill=white,fill opacity=0.5,inner sep=15pt]{\strut\makebox[22cm]{}};
%\draw[anchor=west] (\Gm@lmargin+.3cm,-9cm) node {\huge\sffamily\bfseries\color{black}\thechapter\autodot~#1\strut};
%\end{tikzpicture}};
%\end{tikzpicture}
%\else
%\begin{tikzpicture}[remember picture,overlay]
%\node at (current page.north west)
%{\begin{tikzpicture}[remember picture,overlay]
%\node[anchor=north west,inner sep=0pt] at (0,0) {\ifusechapterimage\includegraphics[width=\paperwidth]{\thechapterimage}\fi};
%\draw[anchor=west] (\Gm@lmargin,-9cm) node [line width=2pt,rounded corners=15pt,draw=ocre,fill=white,fill opacity=0.5,inner sep=15pt]{\strut\makebox[22cm]{}};
%\draw[anchor=west] (\Gm@lmargin+.3cm,-9cm) node {\huge\sffamily\bfseries\color{black}#1\strut};
%\end{tikzpicture}};
%\end{tikzpicture}
%\fi\fi\par\vspace*{270\p@}}}

%-------------------------------------------

%\def\@makeschapterhead#1{%
%\begin{tikzpicture}[remember picture,overlay]
%\node at (current page.north west)
%{\begin{tikzpicture}[remember picture,overlay]
%\node[anchor=north west,inner sep=0pt] at (0,0) {\ifusechapterimage\includegraphics[width=\paperwidth]{\thechapterimage}\fi};
%\draw[anchor=west] (\Gm@lmargin,-9cm) node [line width=2pt,rounded corners=15pt,draw=ocre,fill=white,fill opacity=0.5,inner sep=15pt]{\strut\makebox[22cm]{}};
%\draw[anchor=west] (\Gm@lmargin+.3cm,-9cm) node {\huge\sffamily\bfseries\color{black}#1\strut};
%\end{tikzpicture}};
%\end{tikzpicture}
%\par\vspace*{270\p@}}
%\makeatother

\addto\captionsenglish{\renewcommand{\chaptername}{\sffamily Unit}}

%----------------------------------------------------------------------------------------
%	LINKS
%----------------------------------------------------------------------------------------

\usepackage{hyperref}
\hypersetup{hidelinks,backref=true,pagebackref=true,hyperindex=true,colorlinks=false,breaklinks=true,urlcolor=ocre,bookmarks=true,bookmarksopen=false}

\usepackage{bookmark}
\bookmarksetup{
open,
numbered,
addtohook={%
\ifnum\bookmarkget{level}=0 % chapter
\bookmarksetup{bold}%
\fi
\ifnum\bookmarkget{level}=-1 % part
\bookmarksetup{color=ocre,bold}%
\fi
}
}
 
 
\begin{document}
\pagenumbering{gobble}
\begin{center}
\phantom{xxx} \vspace{7cm} \\
{\LARGE \sc lecture notes on} \\[8pt]
{\Huge \sc \bf Lebesgue Measure Theory} \\
\HRule \\ MATH 422 \\[1cm]
{\sc academic year} \\[3pt] {\Large 2019 - 2020} \\[5cm]
\end{center}
{\large Lectures by \hfill Typeset in \LaTeX, \\[2pt]
{ \bf Dr. I. Subramania Pillai},  \hfill {\bf Edilebert R}, \\[2pt]
Department of Mathematics, \hfill Department of Mathematics, \\[2pt]
Pondicherry University \hfill Pondicherry University}
\newpage
\begin{center}
A significant portion of this text has been adapted from the written notes of {\bf Karthika}, {\bf Neethu} {\bf Edilebert R.}, from a course given by {\bf Dr. I. Subramania Pillai} in Pondicherry University, 2019-2020.
\end{center}
\tableofcontents
\pagenumbering{arabic}
\chapter{Algebras of Sets, and Measures}
\Section{Closed intervals in the Real line}{December 11, 2019}
The set $[a,b]$ holds a special value in real analysis. This set is compact in $\langle \Real, \text{distance} \rangle$.
\begin{remark}
Let $E \subset [a,b]$ be a countably infinite set. Then there exists a $f: [a,b] \longrightarrow \Real$ such that \begin{enumerate}
\item $f$ is discontinuous at every point of $E$
\item $f$ is continuous at every point of $[a,b] \setminus E$
\end{enumerate}
\end{remark}
\begin{proof}
Let $E = \lbrace x_1, x_2, \cdots \rbrace \in [a,b]$. \\ \\
Suppose $x_1 < x_2 < \cdots $ \\\\
Let $c_1, c_2, \cdots$ be the jumps. \\\\
If $0 \leq f \leq M$, choose $c_i$'s such that $\displaystyle \sum_{i>0} c_i \leq M$. \\\\
{\bf General case: } \\\\
$\displaystyle f(x) = \sum _{x_i \leq x} c_i$
\end{proof}
\newpage
\Section{Algebras of Sets}{December 2019}
\begin{definition}
An algebra $\sa$ of a non-empty set $X$ is a collection of subsets of $X$ which holds the following properties:
\begin{enumerate}
\item $A^c \in \sa\ \forall\ A\in \sa$
\item $A \cup B \in \sa\ \forall\ A,B \in \sa$
\end{enumerate}
i.e.,
\[ \sa = \qty{A \in \power{X} \mid A \text{ is closed under complement, } A \text{ is closed under finite union}} \]
\end{definition}

\Section{Set Functions}{December 2019}
\subsection{The Extended Real Number System}
\begin{definition}
The extended system of Real Numbers are defined by $\overline \Real = \Real \cup \qty{-\infty, +\infty}$
\end{definition}
\Section{The Lebesgue Outer Measure}{February, 2020}
Now, we are ready to define the following:
\begin{definition}
The {\bf Lebesgue Outer Measure Function} is the function $m^*\colon \power{\Real} \longrightarrow [0, +\infty]$ defined by \[ m^*(E) = \inf \qty{\sum_{k=1}^\infty \ell(I_k)\biggm| E \subseteq \bigcup_{k=1}^{\infty} I_k} \] for all sets $E \in \power{\Real}$.
\end{definition}
\begin{theorem}
The Lebesgue outer measure is translation invariant:
\[ m^* (E + a) = m^*(E)\quad \forall\, a \in \Real\]
\end{theorem}
\begin{theorem}
The Lebesgue outer measure of singleton set is zero.
\begin{proof}
Let $E = \qty{x}$.
\end{proof}
\end{theorem}
\begin{theorem}
The Lebesgue outer measure of a finite set is zero.
\begin{proof}
Let $E \in \power{\Real}$ be a finite set, say $E = \qty{x_1, x_2, \cdots, x_n}$.
\end{proof}
\end{theorem}
\begin{theorem}
The Lebesgue outer measure of a countable set is zero.
\begin{proof}
Let $E \in \power{\Real}$ be a countable set. Define $I_n = \qty[a_n - \dfrac{\epsilon}{2^{n+2}},\quad a_n + \dfrac{\epsilon}{2^{n+2}}]$
\end{proof}
\end{theorem}
\noindent Now, we would like to show that $m^*$ is actually an extension of $\ell$ to the subsets of $\Real$. 
\begin{theorem}
The Lebesgue outermeasure of an interval is its length. \hfill(12 marks)
\begin{proof}
Let $I$ be an interval. \\\\
{\bf Case 1}: $I$ is closed and bounded \\\\
Let $I = [a,b]$, $a,b \in \Real$. Let $\qty{I_n}_{n=1}^{\infty}$ be an open interval cover of $I$
\[ I = [a,b] \subseteq \bigcup_{n=1}^\infty I_n\]
\begin{itemize}
\item \claim $m^*([a,b]) \leq \ell([a,b])$ \\\\
Use the open interval cover $\qty(a- \dfrac{\epsilon}{2},\ b+\dfrac{\epsilon}{2} )$. \[m^*([a,b]) \leq \ell([a,b]) + \epsilon \quad \forall\ \epsilon>0\]
\item \claim $m^*([a,b]) \geq \ell([a,b])$ \\\\
Use the open interval cover $\displaystyle \bigcup_{n=1}^\infty I_n = [a,b]$. $[a,b]$ is compact $\Rightarrow \exists$ a finite sub cover $\displaystyle \bigcup_{n=1}^k I_n = [a,b]$.
\end{itemize}
{\bf Case 2}: $I$ is open/half-open and bounded \\\\
W.L.O.G., Let $I = (a,b)$. $\overline{I} = [a,b]$.
\begin{itemize}
\item $I \subset \overline{I} \Longrightarrow m^*(I) \leq m^*(\overline{I}) = \ell(I)$.
\item $\ell(I)<m^*(I)$
\end{itemize}
\end{proof}
{\bf Case 3}: $I$ is unbounded \\\\
$\ell(I) = +\infty$. 
\end{theorem}



\Section{Countable Sub-additivity of Lebesgue Outer Measure}{February, 2020}

\Section{The notion of Measure Functions}{January 29, 2020}
Unfortunately, there exists no non-negative, countably additive set function $\mu : \mathcal{P}(\Real) \longrightarrow [0,+\infty]$ that preserves the length of intervals (Ulam). 
\begin{exampleT}
Fix $x_0 \in \Real$. Define $\mu_{x_0}:\power{\Real} \longrightarrow [0, +\infty]$ as $\mu_{x_0}(A) = \begin{cases} 1 & \If x_0 \in A \\  0 & \text{ if } x_0 \notin A \end{cases}$\\\\
$\mu_{x_0}$ is countably additive, but does not preserve the length of intervals, as $\mu_{x_0} (I) = 1$ if $I = (x_0-5, x_0+5)$.
\end{exampleT}
\begin{exampleT}
Define $\mu:\power{\Real} \longrightarrow [0, +\infty]$ as $\mu(A) = \begin{cases} \ell(A) & \If A \text{ is an interval } \\  0 & \text{ otherwise} \end{cases}$\\\\
The image of an interval under $\mu$ is the length of the interval. So,  $\mu((3,8)) = \ell(3,8) = 8-3 = 5$. However, $\mu ((3,8) \cup \qty{9}) = 0$, as $(3,8) \cup \qty{9}$ is not an interval.
\end{exampleT}
\noindent It is not possible to extend the length of intervals to $\power{\Real}$. Hence, we seek to find a class of subsets to which we can extend $\ell$ to.\\
\begin{figure}[h]
\centering
\begin{tikzpicture}
\node[right = 0pt of {(-5,2)}, fill=gray!10] {$m^*: \power{\Real} \longrightarrow [0, +\infty]$};
\node[right = 0pt of {(-5,0)}, fill=gray!10] {$m^*\biggm|_\mathcal{M}: \mathcal{M} \longrightarrow [0, +\infty]$};
\node[right = 0pt of {(-5,-2)}, fill=gray!10] {$\ell: \mathcal{I} \longrightarrow [0, +\infty]$};
\node[right = 2pt of {(-1,2)}, fill=gray!10] {By {\bf S. Ulam}, $m^*$ cannot be countably additive};
\node[right = 2pt of {(-1,0)}, fill=gray!10] {We identify a class of sets $\mathcal{M}$, and $m^*\biggm|_\mathcal{M}$ is the required function};
\node[right = 2pt of {(-1,-2)}, fill=gray!10] {We wish to extend the interval-length function to any set};
\draw plot [->, smooth, tension = 1] coordinates{(-5.2,-1.8)  (-5.5, 0)  (-5.2,1.8)};
\node[left = 1pt of {(-5.5,0)}] {\footnotesize extend};
\end{tikzpicture}
\end{figure}\\
$m = m* \bigm|_\mathcal{M}$ is the required function. \\\\
The set $\mathcal{M} \subseteq \power{\Real}$ is expected to have the following properties:
\begin{enumerate}
\item Contains $\Real$
\item Closed under arbitrary union
\item Closed under complement
\end{enumerate}
$\mathcal{M}$ is a $\sigma-$algebra.
\Section{Construction of Measure Functions}{January 30, 2020}
Now we will develop our way to define the properties of measure functions
\begin{theorem}
Let $X$ be a non-empty set and $\sa$ be a $\sigma-$algebra on the subsets of $X$. If $\qty{A_n}_{n=1}^\infty$ is a sequence of subsets from $\sa$, then $\exists$ $\qty{B_n}_{n=1}^\infty$ in $\sa$ :
\begin{enumerate}
\item $B_n \subseteq A_n\quad \forall\ n \in \mathbb{N}$
\item $B_i \cap B_j = \phi\quad \forall\ i \neq j$
\item $\displaystyle \bigcup_{n=1}^\infty A_n= \bigcup_{n=1}^
\infty B_n$
\end{enumerate}
\begin{proof}
Start with a sequence $\qty{A_n}_{n=1}^\infty$ in $\sa$. Now,
\begin{align*}
\text{For } A_1, &\text{ take } B_1 = A_1 \\
\text{For } A_2, &\text{ take } B_2 = A_2 \setminus A_1 \\
\text{For } A_3,	 &\text{ take } B_3 = A_3 \setminus (A_1 \cup A_2) \\
&\vdots \\
\text{For } A_n,	 & \text{ take } B_n = A_n \setminus \left(\displaystyle \bigcup_{i=1}^{n-1} A_i\right) \\
\end{align*}
The constructed sequence $\qty{B_n}_{n=1}^\infty$ satisfies all the three properties
\end{proof}
\end{theorem} 
\section{Measure Functions and Measure Spaces}
\begin{definition}
Let $X$ be a non-empty set, $\sa$ be a $\sigma-$algebra on the subsets of $X$ and $\mu: \sa \longrightarrow [0, +\infty]$ be a set function such that $\mu(\phi) = 0$. We say
\begin{enumerate}
\item $\displaystyle \mu$ is monotone if $\mu(A) \leq \mu(B)\ \forall\ A \subseteq B$
\item $\mu$ is finitely subadditive if $\displaystyle \mu \left( \bigcup_{i=1}^n A_i \right) \leq \sum_{i=1}^n \mu(A_i)\quad \forall\ \qty{A_n}_{n=1}^\infty \subset \sa$
\item $\mu$ is finitely additive if $\displaystyle \mu \left( \bigcup_{i=1}^\infty A_i \right) = \sum_{i=1}^\infty \mu(A_i)\quad \forall\ \qty{A_n}_{n=1}^\infty \subset \sa$
\item $\mu$ is countably subadditive if $\displaystyle \mu \left( \bigcup_{i=1}^\infty A_i \right) \leq \sum_{i=1}^\infty \mu(A_i)\quad \forall\ \qty{A_n}_{n=1}^\infty \subset \sa$
\item $\mu$ is countably additive if $\displaystyle \mu \left( \bigcup_{i=1}^\infty A_i \right) = \sum_{i=1}^\infty \mu(A_i)\quad \forall\ \qty{A_n}_{n=1}^\infty \subset \sa$
\end{enumerate}
\end{definition}
The study of the relation between all the above definitions will be useful.
\begin{theorem}
Let $X$ be a non-empty set, $\sa$ be a $\sigma-$algebra on the subsets of $X$ and $\mu: \sa \longrightarrow [0, +\infty]$ be a set function such that $\mu(\phi) = 0$. Then,
\begin{enumerate}
\item $\mu$ is finitely additive $\Rightarrow$ $\mu$ is monotone
\item $\mu$ is countably additive $\Leftrightarrow$ $\mu$ is finitely additive and countably subadditive
\end{enumerate}
\begin{proof}
1. Let $A \subseteq B \in \sa$
\end{proof}
\end{theorem}
Now, we are ready to define the {measure} on a $\sigma-$algebra:
\begin{definition}[Measure Function]
Let $X$ be a non-empty set and $\sa$ be a $\sigma-$algebra on the subsets of $X$. Then, the set function \[\mu: \sa \longrightarrow [0, +\infty]\] is called a measure on $\sa$ when $\mu$ is countably additive and $\mu(\phi) = 0$.
\end{definition}
\noindent Equipping a set with a $\sigma-$algebra on its subsets and a measure function will create a measure space:
\begin{definition}[Measure Space]
Let $X$ be a non-empty set, $\sa$ be a $\sigma-$algebra on the subsets of $X$ and $\mu: \sa \longrightarrow [0, +\infty]$ be a measure function. Then the triplet $(X, \sa, \mu)$ is called a measure space.
\end{definition}
\begin{exampleT}
The triplet $(\Real, \sm, m)$ is a measure space, where \\\\$\sm$ is the $\sigma-$algebra on real subsets, called as {\em Lebesgue measurable sets} and\\\\ $m = m^*\bigm|_\sm$ is called the {\em Lebesgue measure}, where $\displaystyle m^*(A) = \inf \qty{\sum_{k=1}^\infty \ell(I_k)\biggm| A \subseteq \bigcup_{k=1}^{\infty} I_k}$.
\end{exampleT}
\begin{exampleT}
In probability theory, the triplet $(\Omega, \mathcal{F}, P)$ is a measure space, where \\\\ $\Omega$ is the set of all possible outcomes (sample space), \\\\ $\mathcal{F}$ is the set of all events considered, and\\\\$P:\mathcal{F} \longrightarrow [0,1]$ is the probability measure. This space is called {\em probability space}.
\end{exampleT}
\begin{definition}
A measure space is said to be complete whenever every subset of every null set is measurable.
\end{definition}
\Section{Lebesgue Measurable Sets and the Lebesgue Measure}{February, 2020}
We have seen that the Lebesgue outer measure $m^*$ is countably additive on a $\sigma$-algbera of $\Real$. We proceed to call the sets of the $\sigma$-algebra as {\em Lebesgue measurable sets} of $\Real$. However, any $\sigma$-algebra of $R$ will make $m^*$ countably additive. Hence, we begin the construction of the ``set of all Lebesgue measurable functions" by defining the following criterion for a set to be ``measurable"
\subsection{Carath\'eodory's Criterion}
\begin{definition}[Carath\' eodory's Criterion]
Let $E$ be a non-empty subset of $\Real$. We say $E$ satisfies Carath\'eodory's criterion when 
\[ m^*(A) = m^*(A \cap E) + m^*(A \cap E^c)\quad \forall\ A \subseteq \Real \]
The collection of all sets which satisfies the Carath\'eodory's criterion is denoted by $\sm$. 
\[ \sm = \qty{E \in \Real \mid \quad m^*(A \cap E) + m^*(A \cap E^c) = m^*(A) \quad \forall\ A \subseteq \Real} \]
\end{definition}
\subsection{Properties of $\sm$}
\begin{theorem}
$E \in \sm$ if and only if $E^c \in \sm$
\end{theorem}
\begin{theorem}
$\phi$ and $\Real \in \sm$ 
\end{theorem}
\begin{theorem}
If $m^*(E) = 0$, then $E \in \sm$ 
\begin{proof}
easy.
\end{proof}
\end{theorem}
\begin{theorem}
If $E_1$ and $E_2 \in \sm,$ then $E_1 \cup E_2 \in \sm$
\end{theorem}
\begin{theorem}
$\sm$ is a $\sigma$-algebra on $\Real$.
\end{theorem}
\begin{theorem}
The triplet $(\mathbb{R}, \sm, m)$ is a complete measure space.
\end{theorem}
\begin{definition}
The measure function $m = m^*|_{\sm}\colon \sm \longrightarrow [0, +\infty]$ is called the Lebesgue measure.
\end{definition}
\subsection{Continuity Properties of Lebesgue Measure}
\begin{theorem}
[First Continuity Property]
Let $\qty{A_n}_{n=1}^\infty$ be a collection of Lebesgue measurable sets such that $A_n \subseteq A_{n+1}\ \forall\ n\in \mathbb{N}$. Then, \[m\qty(\bigcup_{n=1}^\infty A_n) = \lim_{n \to \infty} m(A_n)\]
\begin{proof} 
We begin by dividing the proof into two cases:\\\\
{\bf Case 1}: The sequence contains a set of infinte measure \\\\
{\bf Case 2}: All the sets are of finite measure
\end{proof}
\end{theorem}
\begin{theorem}[Second Continuity Property]
Let $\qty{A_n}_{n=1}^\infty$ be a collection of Lebesgue measurable sets such that $A_{n+1} \subseteq A_{n}\ \forall\ n\in \mathbb{N}$ and $m(A_n)<+\infty\ \forall\ n\in\mathbb{N}$. Then, \[m\qty(\bigcap_{n=1}^\infty A_n) = \lim_{n \to \infty} m(A_n)\]
\begin{proof} 
This can be proved using de Morgan's law on the previous theorem.
\end{proof}
\end{theorem}
\subsection{Regularity Properties of Lebesgue Measure}
\begin{theorem}
Let $E \subset \Real$. Prove that the following are equivalent:
\begin{enumerate}
\item $E$ is measurable
\item $\forall\ \epsilon>0\ \exists \text{ open } G \supset E : m^*(G \setminus E) <  \epsilon$.
\item $\exists$ a $\sG_\delta$ - set $G \supset E : m^*(G \setminus E) = 0$. 
\end{enumerate}
\end{theorem}
\begin{theorem}
Let $E \subset \Real$. Prove that the following are equivalent:
\begin{enumerate}
\item $E$ is measurable
\item $\forall\ \epsilon>0\ \exists \text{ closed } F \subset E : m^*(E \setminus F) <  \epsilon$.
\item $\exists$ a $\sF_\sigma$ - set $F \subset E : m^*(E \setminus F) = 0$. 
\end{enumerate}
\end{theorem}
\newpage
\Section{The Cantor Set}{December 11, 2019}
\vspace{-10pt}
\epigraph{\em ``97.3 percent of all counter examples in Real Analysis involve the Cantor set"}{Unknown}
\noindent We begin with the interval $C_0 = [0,1]$. \\
\begin{figure}[h]
\centering
\begin{tikzpicture}
\node[above = 3pt of {(-8,0)}] {$0$};
\node[above = 3pt of {(8,0)}] {$1$};
\draw[{Bracket[]}-{Bracket[]}, scale = 2] (-4,0) -- (4,0);
\end{tikzpicture}
\end{figure} \\
We split the interval into three equal halves and remove the middle open interval. The remaining set is 
\begin{align*}
C_1 &= C_0 \setminus \qty(\tfrac13,\tfrac23) \\[5pt]
&= \qty[0,\frac13] \cup \qty[\frac23, 1]
\end{align*}
\begin{figure}[h!]
\centering
\begin{tikzpicture}
\node[above = 3pt of {(-8,0)}] {$0$};
\node[above = 3pt of {(8,0)}] {$1$};
\node[above = 3pt of {(-3,0)}] {$\dfrac13$};
\node[above = 3pt of {(3,0)}] {$\dfrac23$};
%\foreach \x in {-3,3}
%\draw[shift={(\x,0)},color=black] (0pt,3pt) -- (0pt,-3pt);
\draw[{Bracket[]}-{Bracket[]}] (-8,0) -- (-3,0);
\draw[{Bracket[]}-{Bracket[]}] (3,0) -- (8,0);
\end{tikzpicture}
\end{figure} \\
Now, we split each of the disjoint intervals into three parts and remove their respective middle open intervals.
The remaining set is 
\begin{align*}
C_2 &= C_1 \setminus \ \qty[\qty(\tfrac19, \tfrac29) \cup \qty(\tfrac79, \tfrac89)] \\[5pt]
&= \qty[0, \frac19] \cup \qty[\frac29, \frac39] \cup \qty[\frac69, \frac79] \cup \qty[\frac89, 1]
\end{align*}
\begin{figure}[h!]
\centering
\begin{tikzpicture}
\node[above = 3pt of {(-8,0)}] {$0$};
\node[above = 3pt of {(-6.3,0)}] {$\dfrac19$};
\node[above = 3pt of {(-4.7,0)}] {$\dfrac29$};
\node[above = 3pt of {(-3,0)}] {$\dfrac39$};
\node[above = 3pt of {(3,0)}] {$\dfrac69$};
\node[above = 3pt of {(4.7,0)}] {$\dfrac79$};
\node[above = 3pt of {(6.3,0)}] {$\dfrac89$};
\node[above = 3pt of {(8,0)}] {$1$};
\draw[thick, dotted] (-7.15, -0.5) -- (-7.15, -1.5);
\draw[thick, dotted] (-3.85, -0.5) -- (-3.85, -1.5);
\draw[thick, dotted] (7.15, -0.5) -- (7.15, -1.5);
\draw[thick, dotted] (3.85, -0.5) -- (3.85, -1.5);
%\foreach \x in {-3,3}
%\draw[shift={(\x,0)},color=black] (0pt,3pt) -- (0pt,-3pt);
\draw[{Bracket[]}-{Bracket[]}] (-8,0) -- (-6.3,0);
\draw[{Bracket[]}-{Bracket[]}] (-4.7,0) -- (-3,0);
\draw[{Bracket[]}-{Bracket[]}] (3,0) -- (4.7,0);
\draw[{Bracket[]}-{Bracket[]}] (6.3,0) -- (8,0);
\end{tikzpicture}
\end{figure} \\
Proceeding further, we obtain an infinite series of closed sets 
\begin{align*}
C_0 \supset C_1 \supset C_2 \supset \cdots \supset C_n \supset \cdots
\end{align*}
\begin{definition}
The Cantor set is given by $\displaystyle \mathscr{C} = C_\infty =\bigcap_{n=1}^\infty C_n$.
\end{definition}
We have shown that any countable set has Lebesgue measure zero. The converse is not true. The Cantor set stands as a counter example:
\begin{theorem}
The Cantor set has Lebesgue measure zero.
\begin{proof}
The collection $\qty{C_n}_{n=1}^\infty$ 
\begin{align*}
C_0 &= [0,1] \\
C_1 &= \qty[0,\frac13] \cup \qty[\frac23, 1] \\
C_2 &= \qty[0,\frac19] \cup \qty[\frac29, \frac39] \cup \qty[\frac69, \frac79] \cup \qty[\frac89, 1] \\
&\vdots
\end{align*}
is measurable, and $m(C_n) = \qty(\dfrac{2}{3})^n\ \forall\ n\in \mathbb{N}$. \\\\
By second continuity property, we can show that
\begin{align*}
m(\mathscr{C}) &= m \qty(\bigcap_{n=1}^\infty C_n)\\
              &= \lim_{n \to \infty} m(C_n) \\
              &= \lim_{n \to \infty} \qty(\dfrac{2}{3})^n = 0
\end{align*}
\end{proof}
\noindent {\bf Aliter:}
\begin{proof}
For each $n$, let $E_n$ be the union of the intervals removed in the $n^\mathrm{th}$ step. Then,
\begin{align*}
m(E_n) &= \dfrac{1}{2} \qty(\dfrac{2}{3})^n \\
\intertext{The total measure of the intervals removed from [0,1] are}
m\qty(\bigcup_{n=1}^\infty E_n) &=  \dfrac{1}{2} \sum_{n=1}^\infty \qty(\dfrac{2}{3})^n = \dfrac12 \cdot \dfrac{\frac23}{1 - \frac23} = 1 \\\\
\therefore\ m(\mathscr{C}) &= m([0,1]) - m\qty(\bigcup_{n=1}^\infty E_n) \\
&= 1 - 1 = 0
\end{align*}
\end{proof}
\end{theorem}
\begin{theorem}
The Cantor set is uncountable.
\begin{proof}
Suppose $\mathscr{C}$ is countable, i.e., $\mathscr{C} = \qty{c_n}_{n=1}^\infty$. 
\begin{itemize}
\item The point $c_1$ is contained in one of the intervals in the union $C_1$. We choose the interval $F_1$ which does not contain $c_1$.
\item Furthermore, at least one of the two intervals in $C_2$ whose union is $F_1$ does not contain $c_2$. We choose the interval $F_2$ which does not contain $c_2$.
\item Proceeding further, we have a nested sequence $\qty{F_n}_{n=1}^\infty$ which are closed and bounded.
\item By nested interval theorem, $\displaystyle \exists\ c : \bigcap_{n=1}^\infty F_n = \qty{c}$. 
\item But, by the construction, this should be empty 
\item Therefore, $\mathscr{C}$ is uncountable.
\end{itemize}
\end{proof}
\end{theorem}
\chapter{Lebesgue Measurable Functions}
\Section{Construction of Non-Lebesgue Measurable subset of [0,1]}{January 2020}
All the general subsets of $\Real$ are measurable so far. We now proceed to show the existence of a subset of $\mathbb{R }$ which is non-measurable.
\begin{theorem}[Vitali]
There exists a Non-Lebesgue measurable set.
\begin{proof}
Define the relation $\sim$ on [0,1] where \[ x \sim y \text{ means } x - y \in \mathbb{Q}\quad \forall\ x,y \in [0,1] \]
We observe the following:
\begin{itemize}
\item $x-x = 0 \in \mathbb{Q} \Rightarrow$ \fbox{$x\sim x$}
\item $x-y = 0 \in \mathbb{Q} \Rightarrow y-x \in \mathbb{Q}.$ So, \fbox{$x \sim y \Rightarrow y \sim x$}
\item $x - y \in \mathbb{Q}$ and $y - z \in \mathbb{Q} \Rightarrow (x-y) + (y-z) \in \mathbb{Q} \Rightarrow x-z \in \mathbb{Q}$. So, \fbox{$x \sim y$ and $y \sim z \Rightarrow x \sim z$}
\end{itemize}
Therefore, $\sim$ is an equivalence relation on $[0,1]$, hence, $\sim$ partitions [0,1] in to equivalence classes of $\sim$. \\
{\bf Define} $V$ as the set of all representatives from each $\sim$-partition of [0,1]. Note that $V \subseteq [0,1]$. \\ \\ Let $\qty{r_1,\ r_2,\ \cdots}$ be an enumeration of $\mathbb{Q} \cap [-1,1]$. Consider the collection \[ \qty{V_i}_{n=1}^{\infty} \] where  $V_i=V + r_i = \qty{v + r_i \mid v\in V}$. We now make the following claims:
\begin{itemize}
\item \claim \fbox{All $V_i$'s are pairwise disjoint} \\ \\
Suppose not, i.e., $\exists\ x \in (V+r_i) \cap (V + r_j)$ for some $i\neq j$. \\ Then, $x$ can be written as both $x = a + r_i$ and $x = b + r_j$ as well, for some $a,b \in V$. \\
$x = a + r_i = b + r_j \Longrightarrow a-b = r_j - r_i$. \\
But $r_j - r_i \in \mathbb{Q} \Longrightarrow a-b \in \mathbb{Q} \Longrightarrow a \sim b$ \\
$\Longrightarrow \Longleftarrow V$ has only one representative from each class. \\ Hence, $V_i$'s are pairwise disjoint. 
\item \claim \fbox{$\displaystyle [0,1] \subseteq \bigcup_{k=1}^{\infty} V_k \subseteq [-1,2]$} \\ \\
$V \subseteq [0,1] \Longrightarrow V + r_i \subseteq [-1,2]\ \forall\ r_i \in [-1,1]$. Hence, $\displaystyle V_i \subseteq [-1,2] \Longrightarrow \bigcup_{k=1}^\infty V_k \subseteq [-1,2]$.\\ \\
Let $x \in [0,1]$. Then $x$ must contain in some equivalence class of $\sim$, i.e., $\exists\ v \in V : x \in v + r_k$ for some $r_k \in \mathbb{Q}$. Hence, $\displaystyle x \in V + r_k \Longrightarrow [0,1] \subseteq \bigcup_{k=1}^\infty V_k$.
\end{itemize}
\claim \fbox{$V$ is not Lebesgue measurable.} \\ \\
Suppose $V$ is Lebesgue measurable. Then all $V_i$'s are Lebesgue measurable.
\begin{itemize}
\item Applying Lebesgue measure using $\sigma$-additivity, we have $\displaystyle 1 \leq \sum_{k=1}^{\infty} m(V_k) \leq 3$
\item Since $V$ is translation invariant, we have $\displaystyle 1 \leq \sum_{k=1}^{\infty} m(V) \leq 3$ \[\Longrightarrow \Longleftarrow m(V) = \begin{cases} 0 & \text{if } m(V) < 1 \\ +\infty & \text{if } m(V) \geq 1 \end{cases}\] i.e., $m(V)$ is a constant $<1$ or $\geq 1$, and the infinite-sum will be 0 or $+\infty$ respectively.
\end{itemize}
Hence, $V$ cannot be Lebesgue measurable.
\end{proof}
\end{theorem}
\noindent The set $V$ is called a {\bf Vitali set}, named after {\bf Giuseppe Vitali}.
\Section{Measurable Functions}{February, 2020}
\begin{definition}
Let $E$ be a Lebesgue measurable function. Then, the function $f\colon E \longrightarrow \overline{\Real}$ is said to be {\bf \em Lebesgue measurable} when the set \[ \qty{x\in E \mid f(x)<\alpha} \text{ is measurable}\ \forall\ \alpha\in \Real\]
\end{definition}
\noindent The following theorem shows that the above criteria can be stated in other ways.
\begin{theorem}
Let $f:E\longrightarrow\overline{\Real}$ be a function where $E$ is measurable. Then, the following are equivalent: 
\begin{enumerate}
\item the set $\qty{x\in E \mid f(x)<\alpha} \text{ is measurable}\ \forall\ \alpha\in \Real$
\item the set $\qty{x\in E \mid f(x)\geq\alpha} \text{ is measurable}\ \forall\ \alpha\in \Real$
\item the set $\qty{x\in E \mid f(x)>\alpha} \text{ is measurable}\ \forall\ \alpha\in \Real$
\item the set $\qty{x\in E \mid f(x)\leq\alpha} \text{ is measurable}\ \forall\ \alpha\in \Real$
\item the set $\qty{x\in E \mid f(x)=\alpha} \text{ is measurable}\ \forall\ \alpha\in \Real$
\end{enumerate}
\end{theorem}
\Section{Properties of Measurable Functions}{February, 2020}
We now show that the collection of Lebesgue measurable functions has somewhat of a linear structure to it.
\begin{theorem}[Linearity of Measurable Functions]
Let $f$ and $g$ be Lebesgue measurable functions defined on the measurable set $E$. Then, 
\begin{itemize}
\item $f+g$ is a Lebesgue measurable function on $E$.
\item $cf$ is a Lebesgue measurable function on $E$, where $c \in \Real$.
\end{itemize}
\end{theorem}
\begin{lemma}
Let $f$ be a Lebesgue measurable function defined on the measurable set $E$. Then $f^2$ is a Lebesgue measurable function.
\end{lemma}
\begin{theorem}
Let $f$ and $g$ be Lebesgue measurable functions defined on the measurable set $E$. Then $fg$ is a Lebesgue measurable function.
\end{theorem}
\begin{theorem}
Every continuous real valued function defined on a Lebesgue measurable set is Lebesgue measurable
\begin{proof}
Let $E$ be a Lebesgue measurable set, and $f\colon E \longrightarrow \Real$ be a continuous function. 
\end{proof}
\end{theorem}
\Section{Approximation by Simple Functions}{March 2, 2020}
\Section{Egorov's Theorem}{March 3, 2020}
Consider $f\colon [0,1] \longrightarrow \Real$ defined by $f_n(x) = x^n\ \forall\ x \in [0,1].$ The pointwise limit of this sequence is the function $f\colon[0,1]\longrightarrow \Real$ defined by \[f(x) = \begin{cases} 1 & x=1 \\ 0 & 0\leq x< 1 \end{cases}\]
We have seen that this convergence is not uniform. We fix this by removing the trouble causing point 1. Then $\displaystyle \lim_{n \to \infty} \sup_{[0, 1-\epsilon]} f_n(x) = \lim_{n \to \infty} (1-\epsilon)^n = 0$. We attempt the same procedure of `removing the troubling points' to any function.
\begin{theorem}[Severini $-$ Egorov]
Let $m(E) < +\infty$, and the sequence $f\colon E \longrightarrow \overline{\Real}$ converge to $f\colon E \longrightarrow \Real$ pointwise. Then, $\forall\ \epsilon>0\ \exists \text{ a measurable subset } A \subset E \text{ with } m(A) < \epsilon : f_n \longrightarrow f \text{ converges uniformly on } E\setminus A$.
\begin{proof}
For $n,k \in \mathbb{N}$, we define the set $E_{n,k}$ by the following union:
\begin{align*}
E_{n,k} &= \bigcup_{m\geq n} \qty{x \in A \biggm| |f_m(x) - f(x)| \geq \dfrac{1}{k}} \\
\intertext{by applying $\sigma$-additivity, we conclude}
m(A) &\leq \sum_{k \in \mathbb{N}} m(E_{n_k,k}) < \sum_{k \in \mathbb{N}} \frac{\epsilon}{2^k} = \epsilon.
\end{align*}
\end{proof}
\end{theorem}
\newpage
\section{Previous Year Questions}
\begin{enumerate}
\item Prove that any set of zero outer measure is Lebesgue measurable.
\item Prove that Lebesgue outer measurable sets are translation invariant.
\item If $E_1$ and $E_2$ are Lebesgue measurable, $E_1 \cup E_2$ is Lebesgue outer measurable.
\item (True or False?) Every real continuous function defined on a Lebesgue measurable set is Lebesgue measurable.
\item (True or False?) Every monotone real function on [0,1] is measurable.
\item Show that Lebesgue (outer) measure of the Cantor set is zero.
\item Show that $\mu\colon \sa \longrightarrow [0,+\infty]$ with $\mu(\phi)=0$ is countably additive if and only if $\mu$ is finitely additive and countably sub additive.
\item If $A_1 \subseteq A_2 \subseteq \cdots A_n \subseteq \cdots$ are measurable, then $\displaystyle m\qty(\bigcup^\infty A_n) = \lim_{n \to \infty} m(A_n)$.
\item Let $E \subset \Real$. Prove that the following are equivalent:
\begin{enumerate}
\item $E$ is measurable
\item $\forall\ \epsilon>0\ \exists \text{ open } G \supset E : m^*(G \setminus E) <  \epsilon$.
\item $\exists$ a $\sG_\delta$ - set $G \supset E : m^*(G \setminus E) = 0$. 
\end{enumerate}
\item Let $E \subset \Real$. Prove that the following are equivalent:
\begin{enumerate}
\item $E$ is measurable
\item $\forall\ \epsilon>0\ \exists \text{ closed } F \subset E : m^*(E \setminus F) <  \epsilon$.
\item $\exists$ a $\sF_\sigma$ - set $F \subset E : m^*(E \setminus F) = 0$. 
\end{enumerate}
\item Prove that Lebesgue outer measure of an interval is its length.
\item Construct a non-Lebesgue measurable subset of $[0,1]$. 
\item Define Simple functions and its canonical form.
\item Show that if $f$ and $g$ are Lebesgue measurable functions, then $f+g$ is a Lebesgue measurable function.
\item Prove that the point wise limit of a sequence of measurable functions is measurable.
\item $\forall$ bounded, non-negative $f\colon E \longrightarrow \overline{\Real}\ \exists $ non-negative, simple measurable functions $\qty{f_n}_{n=1}^\infty : f_n(x) \longrightarrow f(x)$ p.w.
\item State and prove Egorov's theorem.
\end{enumerate}
\chapter{The Lebesgue Integral}
\epigraphhead[120]{\em \raggedleft ``I have to pay a certain sum, which I have collected in my pocket. I take the bills and coins out of my pocket and give them to the creditor in the order I find them until I have reached the total sum. This is the Riemann integral. But I can proceed differently. After I have taken all the money out of my pocket I order the bills and coins according to identical values and then I pay the several heaps one after the other to the creditor. This is my integral."}{\raggedleft Henri Lebesgue, {\em } \par}
\Section{Lusin's Theorem}{February 2019}
The Egoroff's theorem is a precise realization on Littlewood's last principle on real (measurable) functions. We now develop upon the theorem to state a more general version:
\begin{theorem} [Lusin's theorem]
Let $f\colon E \longrightarrow \Real$ be measurable. Then, \\ \\
$\forall\ \epsilon>0\ \exists\ F \subset [a,b]$ and a continuous $g : \Real \longrightarrow \Real$ such that
\begin{align*}
\restr{g}{E} = f \\[2pt]
\text{and } m(E \setminus F) < \epsilon
\end{align*}
\begin{proof}
We begin with applying Egorov's theorem.
\end{proof}
\end{theorem}
\Section{The Lebesgue Integral}{March 16, 2020}
Upon proving several theorems, the takeaway are the three principles of Littlewood:
\begin{enumerate}
\item A measurable set can be almost written as an union of open intervals
\item The limit point of an uniformly converging measurable functions is  almost measurable
\item A measurable function is continuous almost everywhere
\end{enumerate}
Now, we are ready to define the {\bf Lebesgue integral}. We begin with simple functions:
\subsection{Lebesgue Integral of a Simple Function}
\begin{theorem}
[Characterization of Simple Functions]
Any simple functions can be `characterized'.
\begin{proof}
$(\Rightarrow)$ Let $\phi : E \longrightarrow \mathbb{R}$ be a simple function, i.e., $\phi$ is measurable and range $\phi$ is finite, say,
\begin{align*}
\qquad\phi(E) &= \qty{a_1,a_2, \cdots, a_n} \\
\intertext{Define $E_i$'s as follows:}
E_i &= \qty{x \in E \mid \phi(x) = a_i}\quad \forall\ 1\leq i \leq n 
\intertext{Note that $\displaystyle {\bigcup_{i=1}^n}\ E_i = E$.  Hence,}
\phi &= \sum_{i=1}^n a_i\,\chi_{E_i}
\end{align*}
\end{proof}
\begin{proof}
$(\Leftarrow)$ Suppose $\phi : E \longrightarrow \mathbb{R}$ is defined as
\begin{align*}
\phi &= \sum_{i=1}^m b_i\,\chi_{A_i}
\intertext{where $A_i$'s are measurable, and $A_i \subset E\ \forall\ 1\leq i\leq n$. Note that $b_i$'s need not be aligned} 
\end{align*}
\end{proof}
\end{theorem}
\begin{definition}
Let $\phi : E \longrightarrow \mathbb{R}$ be a simple function defined on a finite measurable set $E$. If $\phi$ can be expressed in canonical form as 
\begin{align*}
\phi &= \sum_{i=1}^n a_i\, \chi_{E_i} \\
\intertext{Then we define the \underline{Lebesgue integral of $\phi$ over $E$} as} 
\int\displaylimits_E \phi(x) \dd{x} &= \sum_{i=1}^n a_i\,m({E_i}) \tag{ $\forall\, x\in E$ }\\
\intertext {or simply,} 
\int_E \phi &= \sum_{i=1}^n a_i\,m({E_i})
\end{align*}
\end{definition}
The sum in the Lebesgue integral in fact need not be in canonical form. 
\begin{theorem}
Let $\phi : E \longrightarrow \mathbb{R}$ be a simple function  defined on a finite measured set. If $\phi$ is defined as \[\phi = \sum_{n=1}^\infty b_j\,\chi_{E_j}\]
which need not be in the canonical form. Then, \[\int_E \phi = \sum_{i=1}^n b_j\,m({E_j})\]
\begin{proof}
Let 
\end{proof}
\end{theorem}
\chapter{Riemann vs Lebesgue Integration}
\Section{A quick review of the Riemann Integral}{December 12, 2019}
Let $f:[a,b] \longrightarrow \mathbb{R}$. Without loss of generality, keep $f \geq 0$. \\ \\
{\bf Basic problem: } To find the area of the region bounded by the graph of $f$ and the $x$-axis. \\ \\
{\bf Idea: } Cut the region into vertical slices and approximate them into rectangular strips.
\begin{figure}[!h]
\centering

\begin{tikzpicture}
\draw[->] (-1,0) -- (5,0) node[right] {$x$};
\draw[->] (0,-1) -- (0,4) node[above] {$y$};
\draw plot [smooth, tension=0.6] coordinates { (1,2) (2,3) (3,2.5) (4,3.5)};
\draw[scale=1,domain=-0.5:2,variable=\y, densely dashed]  plot ({1},{\y});
\draw[scale=1,domain=-0:2.6,variable=\y, densely dashed]  plot ({1.5},{\y});
\draw[scale=1,domain=-0.:3,variable=\y, densely dashed]  plot ({2},{\y});
\draw[scale=1,domain=-0:2.75,variable=\y, densely dashed]  plot ({2.5},{\y});
\draw[scale=1,domain=-0:2.5,variable=\y, densely dashed]  plot ({3},{\y});
\draw[scale=1,domain=-0:2.9,variable=\y, densely dashed]  plot ({3.5},{\y});
\draw[scale=1,domain=-0.5:3.5,variable=\y, densely dashed]  plot ({4},{\y});
%===VERTICAL LINES - RED===
\draw[densely dashed, blue]  (1,2) -- (1.5,2);
\draw[densely dashed, red]  (1, 2.6) -- (1.5, 2.6);
\draw[densely dashed, red]  (1, 2) -- (1, 2.6);

\draw[densely dashed, blue]  (1.5,2.6) -- (2,2.6);
\draw[densely dashed, red]  (1.5, 3) -- (2, 3);
\draw[densely dashed, red]  (1.5, 2.6) -- (1.5, 3);

\draw[densely dashed, blue]  (2,2.75) -- (2.5,2.75);
\draw[densely dashed, red]  (2, 3.02) -- (2.5, 3.02);
\draw[densely dashed, red]  (2.5, 2.8) -- (2.5, 3);

\draw[densely dashed, blue]  (2.5,2.5) -- (3,2.5);
\draw[densely dashed, red]  (2.5,2.75) -- (3,2.75);
\draw[densely dashed, red]  (3,2.5) -- (3,2.9);

\draw[densely dashed, blue]  (3,2.5) -- (3.5,2.5);
\draw[densely dashed, red]  (3,2.9) -- (3.5,2.9);

\draw[densely dashed, blue]  (3.5,2.9) -- (4,2.9);
\draw[densely dashed, red]  (3.5,3.5) -- (4,3.5);
\draw[densely dashed, red]  (3.5,2.9) -- (3.5,3.5);
%===LABELS===
\node[scale=0.8, above = 2pt of {(4,3.5)}] {$f$};    
\node[scale=0.8, below = 1pt of {(1,-0.5)}] {$a = x_0$};  
\node[scale=0.8, below = 5pt of {(1.5,0)}] {$x_1$}; 
\node[scale=0.8, below = 5pt of {(2,0)}] {$x_2$}; 
\node[scale=0.8, below = 5pt of {(2.5,0)}] {$x_3$};  
\node[scale=0.8, below = 5pt of {(3,0)}] {$\cdots$};  
\node[scale=0.8, below = 5pt of {(3.5,0)}] {$x_{n-1}$};  
\node[scale=0.8, below right= 1pt of {(4,-0.5)}] {$x_{n} = b$};  
\end{tikzpicture}
\end{figure}

\noindent Let $\sigma \in \Part{a,b}$ be the partition $\sigma = \lbrace a = x_0,\ x_1,\ \cdots,\ x_n = b \rbrace$ by which the region is split. \\ \\
Let $I_i = [x_{i-1},\ x_i]$ and $\Delta x_i = \ell (I_i) = x_i - x_{i-1}$ \\\\
Take $\displaystyle m_i = \inf_{\displaystyle x \in I_i} f(x)$ and $\displaystyle M_i = \sup_{\displaystyle x \in I_i} f(x)$. \\ \\
The {\em infimum area} of an arbitrary rectangular strip which is cut using the partition $\sigma$, called the {\bf lower sum} is given by 
\begin{align*}
L(f, \sigma) &= m_i\ \Delta x_i 
\end{align*}
The {\em supremum area} of an arbitrary rectangular strip which is cut using the partition $\sigma$, called the {\bf upper sum} is given by 
\begin{align*}
U(f, \sigma) &= M_i\ \Delta x_i 
\end{align*}
We observe the following: 
\begin{itemize}
\item $m_i \leq M_i\quad \forall\ i = 1,2, \cdots, n$
\item $L(f, \sigma) \leq U(f, \sigma)\quad \forall\ \sigma \in \Part{a,b}$
\item $L(f, \sigma) \leq L(f, \tau)$ and $U(f, \sigma) \geq U(f, \tau)\quad \forall\ \sigma \subseteq \tau$
\item $L(f, \sigma) \leq L(f,\ \sigma \cup \tau) \leq U(f,\ \sigma \cup \tau) \leq U(f, \sigma)$ \\ 
\end{itemize}
Our intuition says that the thinner the partition is, the better the approximation will be. If the `area' of the region makes sense, the supremum of the lower sum and the infimum of the upper sum will coincide.
\begin{figure}[h]
\centering
\begin{tikzpicture}
\draw[-] (-7,0) -- (7,0);
\foreach \x in {-7, 0, 7}
\draw[shift={(\x,0)},color=black] (0pt,5pt) -- (0pt,-5pt);
\node[below = 6pt of {(-7,0)}] {$L(f, \sigma)$};
\node[below = 6pt of {(7,0)}] {$U(f, \sigma)$};
%\node[on {(0,0)}] {$\times$};
\filldraw (0,0) circle[radius = 2pt];
\node[above right = 3pt of {(0,0)}] {$\displaystyle \sup_{\sigma \in \Part{a,b}} L(f, \sigma)$};
\node[below left = 3pt of {(0,0)}] {$\displaystyle \inf_{\sigma \in \Part{a,b}} U(f, \sigma)$};
\end{tikzpicture}
\end{figure} \\
\newpage
\noindent Now we are ready to define the following:
\begin{definition}
The function $f$ is said to be Riemann integrable on an interval $[a,b]$ when the supremum of the lower sums over all partitions of $[a,b]$ is equal to the infimum of the upper sums over all the partitions of $[a,b]$, i.e., 
\begin{align*}
\sup_{\sigma \in \Part{a,b}} L(f, \sigma) = \inf_{\sigma \in \Part{a,b}} U(f, \sigma)
\end{align*}
\end{definition}
\noindent We then define the {\em lower integral} of $f$ over $[a,b]$ as \[ \displaystyle \loRint{a}{b} f = \sup_{\sigma \in \Part{a,b}} L(f, \sigma) \]
and the {\em upper integral} of $f$ over $[a,b]$ as \[\displaystyle \upRint{a}{b} f = \inf_{\sigma \in \Part{a,b}} U(f, \sigma)\]
When $f$ is Riemann integrable over $[a,b]$, we write $f \in \Riem{a,b}$ and the integral is given as \[ \int_a^b f = \loRint{a}{b} f = \upRint{a}{b} \]
This is equivalent to saying that the lower and upper sums are arbitrarily close to each other. 
\begin{definition}
$\forall\ \epsilon>0\ \exists\ \sigma \in \Part{a,b}\ : U(f,\sigma) - L(f,\sigma) < \epsilon$
\end{definition}
\noindent The above statement is called the {\bf Cauchy's criterion} for Riemann integrability. 
\SEction{Examples}{}
\begin{exampleT}
A classic example of a function which is {\bf not} Riemann integrable is 
\begin{align*}
f&\colon [0,1] \longrightarrow \Real  \\[5pt]
\text{defined by } f(x) &= \left\lbrace \begin{array}{l l}
1 & \text{if $x$ is rational} \\
0 & \text{if $x$ is irrational}
\end{array} \right.\quad \forall\ x \in [0,1]
\end{align*}
$\forall\ \sigma\in \Part{a,b}$, we have $m_i =0$ and $M_i = 1$. Which results in $L(f, \sigma) = 0$ and $U(f, \sigma) = 1$. \\ \\ Choosing $\epsilon = \frac{1}{2}$ will fail the Cauchy criterion, hence, $f$ is not Riemann integrable. \qed
\end{exampleT}
\begin{exampleT} Define $f\colon [0,1] \longrightarrow \Real$ as
\begin{align*}
f(x) &= \begin{cases}
1 & \text{if } x = \frac{1}{2} \\
0 & \text{otherwise}
\end{cases}\quad \forall\ x \in [0,1]
\end{align*}
Take $\sigma \in \Part{a,b}$. We have $\displaystyle L(f, \sigma) = 0$, hence $\displaystyle \loRint{0}{1} f = 0$. Also, $\displaystyle U(f, \sigma) \geq 0$. \\ \\
Given $\epsilon > 0$, take $\sigma_\epsilon = \qty{0,\quad \frac{1}{2} - \frac{\epsilon}{4},\quad \frac{1}{2} + \frac{\epsilon}{4},\quad 1}$. \\\\ Then, $U(f, \sigma_\epsilon) = 0 \cdot \qty(\frac{1}{2} - \frac{\epsilon}{4}) + 1 \cdot \epsilon + 0 \cdot \qty(\frac{1}{2} - \frac{\epsilon}{4}) = \frac{\epsilon}{2} < \epsilon$. 
\begin{figure}[h]
\centering
\begin{tikzpicture}
\draw[-] (-7,0) -- (7,0);
\foreach \x in {-7, 0, 7}
\draw[shift={(\x,0)},color=black] (0pt,5pt) -- (0pt,-5pt);
\node[below = 6pt of {(-7,0)}] {$0$};
\node[below = 6pt of {(7,0)}] {$1$};
\node[above = 6pt of {(0,0)}] {$\frac{1}{2}$};
\draw[(-), thick] (-2,0) -- (2,0);
\node[above = 6pt of {(-2,0)}] {$-\frac{\epsilon}{4}$};
\node[above = 6pt of {(2,0)}] {$+\frac{\epsilon}{4}$};
\end{tikzpicture}
\end{figure} \\
By Cauchy criterion, $f \in \Riem{a,b}$, and by definition, $\displaystyle \int_0^1 f = 0$. \qed
\end{exampleT}
\begin{exampleT}
Define $f\colon [0,1] \longrightarrow \Real$ as
\begin{align*}
f(x) &= \begin{cases}
2 & \text{if } x = a_1 \\
1 & \text{if } x = a_2 \\
0 & \text{otherwise}
\end{cases}\quad \forall\ x \in [0,1] \quad \text{where}\ a_1, a_2 \in [a,b]
\end{align*}
$f \in \Riem{a,b}$, and $\displaystyle \int_0^1 f= 0$ (why?) 
\end{exampleT}
\begin{exampleT}
We can proceed further like the above examples and define $f\colon [0,1] \longrightarrow \Real$ as
\begin{align*}
f(x) &= \begin{cases}
\alpha_i & \text{if } x = a_i \\
0 & \text{otherwise}
\end{cases}\quad \forall\ x \in [0,1] \quad \text{where}
\begin{array}{l l}
a_1, a_2, \cdots a_n \in [a,b] \\
\alpha_1, \alpha_2, \cdots \alpha_n \in \Real \setminus \qty{0}
\end{array}
\end{align*}
\begin{figure}[h]
\centering
\begin{tikzpicture}
\draw[-] (-6,0) -- (6,0);
\foreach \x in {-6,-4,-2,0,2,4,6}
\draw[shift={(\x,0)},color=black] (0pt,6pt) -- (0pt,-6pt);
\draw[-|] (-4,0) -- (-4,0.7);  \node[above = 1pt of {(-4, 0.7)}] {$\alpha_1$};
\draw[-|] (-2,0) -- (-2,1.4); \node[above = 1pt of {(-2, 1.4)}] {$\alpha_2$};
\draw[-|] (0,0) -- (0,0.9); \node[above = 1pt of {(0, 0.9)}] {$\alpha_3$};
\node[above = 1pt of {(2, 0.8)}] {$\cdots$};
\draw[-|] (4,0) -- (4,1.1);  \node[above = 1pt of {(4, 1.1)}] {$\alpha_n$};
\node[below = 6pt of {(-4,0)}] {$a_1$};
\node[below = 6pt of {(-2,0)}] {$a_2$};
\node[below = 6pt of {(0,0)}] {$a_3$};
\node[below = 6pt of {(2,0)}] {$\cdots$};
\node[below = 6pt of {(4,0)}] {$a_n$};
\end{tikzpicture}
\end{figure} \\
We recall the following theorem:
\begin{theorem}
Let $f,g \in \Riem{a,b}$. Then, $f + g \in \Riem{a,b}$, and \[\int_a^b f+g = \int_a^b f  + \int_a^b g\]
\end{theorem}
\noindent We write our $f$ as 
\begin{align*}
f &= f_1 + f_2 + \cdots + f_n \\ \\
\text{where each}\ f_i &= \begin{cases} 
\alpha_i & \text{if}\ x = a_i \\
0 & \text{otherwise}
\end{cases}
\end{align*}
By previous examples, each $f_i \in \Riem{a,b}$. Hence, $f \in \Riem{a,b}$. \\ \\
Also, each $\displaystyle \int_0^1 f_i = 0$. Hence, $\displaystyle \int_0^1 f = 0$ \qed
\end{exampleT}
\noindent {\bf Observation: } If $f \in \Riem{a,b}$, then redefining a finite number of points does not affect the Riemann integrability and the integral value remains the same. 
\begin{figure}[h]
\centering
\begin{subfigure}{.3\textwidth}
\begin{tikzpicture}
\draw[->] (0,0) -- (4,0) node[right] {$x$};
\draw[->] (0,0) -- (0,3) node[above] {$y$};
\node[right = 1.5pt of {(4, 1.5)}] {$=$};
\node[above right = 1pt of {(3.5,3.5)}] {$f$};
\draw plot [smooth, tension=0.6] coordinates { (0.5,2) (1.5,3) (2.5,2.5) (3.5,3.5)};
\node[above left= 1pt of {(1.5,3)}] {$\quad$};
\draw[scale=1,domain=-0:2,variable=\y]  plot ({0.5},{\y});
\draw[scale=1,domain=-0:3.5,variable=\y]  plot ({3.5},{\y});
\node[below = 1pt of {(1.5,0)}] {$\quad$};
\end{tikzpicture}
\end{subfigure} \begin{subfigure}{.3\textwidth}
\vspace{7.8pt}
\begin{tikzpicture}
\draw[->] (0,0) -- (4,0) node[right] {$x$};
\draw[->] (0,0) -- (0,3) node[above] {$y$};
\node[right = 1.5pt of {(4, 1.5)}] {$+$};
\node[above right = 1pt of {(3.5,3.5)}] {$g$};
\draw plot [smooth, tension=0.6] coordinates { (0.5,2) (1.5,3) (2.5,2.5) (3.5,3.5)};
\draw[scale=1,domain=-0:2,variable=\y]  plot ({0.5},{\y});
\draw[scale=1,domain=-0.:3,variable=\y, densely dotted]  plot ({1.5},{\y});
\draw[scale=1,domain=-0:2.5,variable=\y, densely dotted]  plot ({2.5},{\y});
\node[below = 1pt of {(2, 2.5)}] {$\cdots$};
\draw[scale=1,domain=-0:3.5,variable=\y]  plot ({3.5},{\y});
\node[below = 1pt of {(1.5,0)}] {$a_1$};
\node[below = 1pt of {(2,0)}] {$\cdots$};
\node[below = 1pt of {(2.5,0}] {$a_n$};
\draw[fill = white] (1.5,0) circle[radius = 1.5pt];
\draw[fill = white] (2.5,0) circle[radius = 1.5pt];
\draw[fill = white] (1.5,3) circle[radius = 1.5pt];
\draw[fill = white] (2.5,2.5) circle[radius = 1.5pt];
\node[above left= 1pt of {(1.5,3)}] {$\alpha_1$};
\node[below right= 1pt of {(2.5,2.5)}] {$\alpha_n$};
\end{tikzpicture}
\end{subfigure}
\begin{subfigure}{0.3\textwidth}
\vspace{14pt}
\begin{tikzpicture}
\draw[->] (0,0) -- (4,0) node[right] {$x$};
\draw[->] (0,0) -- (0,3) node[above] {$y$};
\node[above right = 1pt of {(3.5,3.5)}] {$\ $};
\node[above right = 1pt of {(2.5,3)}] {$h$};
\draw[scale=1,domain=-0:3,variable=\y, densely dotted]  plot ({1.5},{\y});
\node[left = 1pt of {(1.5,3)}] {$\alpha_1$};
\draw[scale=1,domain=-0:2.5,variable=\y, densely dotted]  plot ({2.5},{\y});
\node[right = 1pt of {(2.5,2.5)}] {$\alpha_n$};
\node[below = 1pt of {(2, 2.5)}] {$\cdots$};
\node[below = 1pt of {(1.5,0)}] {$a_1$};
\node[below = 1pt of {(2,0)}] {$\cdots$};
\node[below = 1pt of {(2.5,0}] {$a_n$};
\draw[fill = black] (1.5,0) circle[radius = 1.5pt];
\draw[fill = black] (2.5,0) circle[radius = 1.5pt];
\draw[fill = black] (1.5,3) circle[radius = 1.5pt];
\draw[fill = black] (2.5,2.5) circle[radius = 1.5pt];
\end{tikzpicture}
\end{subfigure}
\end{figure} \\
We write $f = g + h$, and $\displaystyle \int_a^b h = 0$, i.e., the integral value is unaffected.
\chapter{Absolutely Continuous Functions}
\Section{}{}
\end{document}